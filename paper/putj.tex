% -----------------------------------------------
% Template for ISMIR Papers
% 2015 version, based on previous ISMIR templates
% -----------------------------------------------

\documentclass{article}
\usepackage{ismir}
\usepackage{url}
\usepackage{cleveref}
\usepackage{cite}
\usepackage{brian}
\usepackage{graphicx}
\usepackage{booktabs}

% Title.
% ------
\title{Pump up the JAMS}


% Single address
% To use with only one author or several with the same address
% ---------------
%\oneauthor
% {Names should be omitted for double-blind reviewing}
% {Affiliations should be omitted for double-blind reviewing}

% Two addresses
% --------------
%\twoauthors
%  {First author} {School \\ Department}
%  {Second author} {Company \\ Address}

% Three addresses
% --------------
\threeauthors
  {First author} {Affiliation1 \\ {\tt author1@ismir.edu}}
  {Second author} {\bf Retain these fake authors in\\\bf submission to preserve the formatting}
  {Third author} {Affiliation3 \\ {\tt author3@ismir.edu}}

% Four addresses
% --------------
%\fourauthors
%  {First author} {Affiliation1 \\ {\tt author1@ismir.edu}}
%  {Second author}{Affiliation2 \\ {\tt author2@ismir.edu}}
%  {Third author} {Affiliation3 \\ {\tt author3@ismir.edu}}
%  {Fourth author} {Affiliation4 \\ {\tt author4@ismir.edu}}

\begin{document}
%
\maketitle
%
\begin{abstract}
Predictive models for music annotation tasks are practically limited by a paucity of
well-annotated training.
In this work, we develop a general framework for augmenting annotation musical datasets,
allowing practitioners to expand training sets in a controlled fashion.
\end{abstract}
%
\section{Introduction}
\label{sec:introduction}

\cite{sturmkiki}

\subsection{Our contributions}

\section{Data augmentation}

\cite{mauch2013audio}

\subsection{Audio deformation}

\subsection{Annotation deformation}

\cite{humphreyjams}

\section{Example application: instrument recognition}

\cite{bittner2014medleydb}

\subsection{Data augmentation}

The data augmentation pipeline consists of four stages:

\begin{description}
    \item[Pitch shift] by $n \in \{-1, 0, +1\}$ semitones
    \item[Time stretch] by a factor of $f \in \{ 0.5, 1.0, 1.5\}$
    \item[Additive noise] under four conditions: no noise,
        subway\footnote{\url{https://www.freesound.org/people/jobro/sounds/112252/}},
        crowded concert hall\footnote{\url{https://www.freesound.org/people/klankbeeld/sounds/171317/}},
        and night-time city noise\footnote{\url{https://www.freesound.org/people/inkhorn/sounds/231870/}}.
        The latter three were mixed with random weights drawn uniformly in $[0.1, 0.4]$.
    \item[Dynamic range compression] under three preset conditions drawn from the {Dolby E}
        standards~\cite{dolbyE}: none, \emph{speech},
        and \emph{music (standard)}.
\end{description}

\subsection{Results}

\section{Conclusion}

% For bibtex users:
\bibliography{refs}

\end{document}
