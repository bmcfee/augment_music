% -----------------------------------------------
% Template for ISMIR Papers
% 2015 version, based on previous ISMIR templates
% -----------------------------------------------

\documentclass{article}
\usepackage{ismir}
\usepackage{url}
\usepackage{cleveref}
\usepackage{cite}
\usepackage{brian}
\usepackage{graphicx}
\usepackage{booktabs}

% Title.
% ------
\title{Pump up the jams: musical data augmentation}


% Single address
% To use with only one author or several with the same address
% ---------------
%\oneauthor
% {Names should be omitted for double-blind reviewing}
% {Affiliations should be omitted for double-blind reviewing}

% Two addresses
% --------------
%\twoauthors
%  {First author} {School \\ Department}
%  {Second author} {Company \\ Address}

% Three addresses
% --------------
\threeauthors
  {First author} {Affiliation1 \\ {\tt author1@ismir.edu}}
  {Second author} {\bf Retain these fake authors in\\\bf submission to preserve the formatting}
  {Third author} {Affiliation3 \\ {\tt author3@ismir.edu}}

% Four addresses
% --------------
%\fourauthors
%  {First author} {Affiliation1 \\ {\tt author1@ismir.edu}}
%  {Second author}{Affiliation2 \\ {\tt author2@ismir.edu}}
%  {Third author} {Affiliation3 \\ {\tt author3@ismir.edu}}
%  {Fourth author} {Affiliation4 \\ {\tt author4@ismir.edu}}

\begin{document}
%
\maketitle
%
\begin{abstract}
Predictive models for music annotation tasks are practically limited by a paucity of
well-annotated training data.
In this work, we develop a general framework for augmenting annotated musical datasets,
allowing practitioners to expand training sets in a controlled fashion.
We investigate the effects of data augmentation on the task of recognizing instruments
in mixed signals.
\end{abstract}
%
\section{Introduction}
\label{sec:introduction}

% Main points to make:
%   1. music is complex, and needs big models
%       a. big models need big data
%       b. big (annotated) data is hard to come by

Musical audio signals contain a wealth of rich, complex, and highly structured
information.  The primary goal of content-based music information retrieval (MIR) is to
analyze, extract, and summarize music recordings in a human-friendly
format, such as semantic tags, chord and melody annotations, or structural boundary
estimations.  To adequately capture and characterize the vast complexity of musical
recordings seems to require large, flexible models.  In short, complex data necessitate
complex models.
By the same token, estimating the parameters of complex statistical models often requires
a large number of samples: big models require big data.

Within the past few years, this same phenomenon of model complexity has been observed 
in the computer vision literature.  Currently, the best-performing models for recognition 
of objects in images exploit two fundamental properties to overcome the difficulty of 
fitting large, complex models: access to large quantities of annotated data, and 
identification of label-invariant transformations~\cite{krizhevsky2012imagenet}.
The benefits of large training collections are obvious, and unfortunately, difficult to 
carry over to most musical annotation tasks due to the complexity of the label space and
need for expert annotators.  However, the idea of generating perturbations of a training
set --- known as \emph{data augmentation} --- can be readily adapted to musical tasks.

%   2. in images: 
%       a. data augmentation has proven useful in vision. 
%       b. general idea: perturb your data such that the features change, but the labels
%       don't
%       c. rotations, reflections, contrast normalization, affine transformations
Conceptually, data augmentation consists of the application of one or more deformations to
a collection of (annotated) training samples.  
The motivation for data augmentation is that a learning algorithm should be less
susceptible to spurious correlations and over-fitting if it is provided with many
observations of an instance which have been perturbed in ways which do not affect its
label.
Some concrete examples of deformations drawn from computer vision include translation, 
rotations, and reflections.  These simple operations are appealing because they typically 
do not affect the target class label. An upside-down image of a cat is still contains cat, 
although the situation may be more complex for concepts which are not reflection-invariant, 
such as in optical character recognition.  Consequently, practitioners must exercise some 
caution when applying data augmentation techniques to ensure that the correct invariances 
are maintained.

In general, deformations apply not only to the observed
stimulus, but its annotations as well.
Continuing the image example, if an image is rotated, then any pixel-wise 
label annotations should be modified accordingly.
This observation opens up several interesting possibilities for musical applications, in
which the target concept space typically exhibits a high degree of structure.
As a simple example, time-stretching an audio track should also result in time-interval
annotations moving accordingly~\cite{mauch2013audio}.  As a more complex example,
pitch-shifting a track may (or may not) result in modifications to pitch contours,
chord labels, or symbolic annotations.

%   3. but data augmentation is more difficult in music than in images
%       a. the output space is much more complex than simple tags
%       b. it's not obvious which variations preserve output structure
%       c. simple example: time-stretching will move annotation boundaries
%       d. complex example: pitch-shifting will deform chord labels

\subsection{Our contributions}
% 1. develop a generic framework for synchronously manipulating audio and annotations
% 2. investigate the effects of simple deformations on the problem of musical instrument recognition
In this work, we describe a software architecture for applying data augmentation to music
information retrieval tasks.  The system is designed to be simple, modular, and
extensible, allowing practitioners to develop custom deformations, and combine multiple
simple deformations together into pipelines which can generate large volumes of reliably
deformed, annotated music data.  The proposed system is built on top of
JAMS~\cite{humphreyjams}, which provides a simple container for accessing and
transporting multiple annotations for a given track.  

As a simple proof of concept, we apply the proposed data augmentation architecture to the
task of recognizing instruments in mixed signals, and demonstrate that even simple
manipulations can provide substantial improvements in model performance.

\section{Related work}

% 1. it's common to engineer systems to attempt to resolve symmetries in the input,
%       eg, chroma features are engineered to be approximately invariant to timbre and octave
%       some authors suppress the 0th mfcc to get loudness invariance
As practitioners, the first step in developing a solution to some MIR task is often to 
design features which discard information which is irrelevant to predicting the target
concept.  For example, chroma features are designed to capture pitch class information
and suppress information derived from timbre, loudness, or octave 
height~\cite{muller2011chroma}.  Similarly, many authors interested in modeling timbre
use Mel-frequency cepstral coefficients (MFCCs) and discard the first component to
achieve invariance to loudness~\cite{pampalk2004matlab}.
While this general strategy makes intuitive, practical sense, it carries certain
limitations.  First, it is not necessarily easy to identify all relevant symmetries the
data: if it was easy, the problem would be essentially solved.  Second, even if such
properties are easy to identify, it may still be difficult to engineer invariant
features.  Finally, even if one does succeed in designing appropriately invariant
features, they may inadvertently discard relevant information in the process.

%   but there are some drawbacks:
%       a. it's not easy to identify *all* relevant symmetries
%       b. even if it was, it might not be easy to engineer an invariant feature
%       c. and you might accidentally discard useful information in the process
%
%   alternatively: 
%       we can use bigger models
%       learn the appropriate invariances from statistics.
%       but this takes a lot of (annotated) data, which we usually don't have
%   
%
As an alternative to custom feature design, some authors advocate learning or optimizing
features directly from the data~\cite{humphrey2012moving}.  
Perhaps not surprisingly, this approach typically requires large model architectures, and 
consequently requires much larger (annotated) data sets than had previously been 
used in MIR research.  
Our goal in this work is to ease the burden of sample complexity, and make 
data-driven models more accessible to the MIR community.

% 1. augmentation is not new, but it hasn't been done systematically.
%   eg, chroma rotation for key-invariance in chord quality or mode
%   synthetic mixtures of clean signals
%   perturbations of the labels ``target smearing''
%
Specific instances of data augmentation can be found throughout the MIR literature,
though they are not often identified as such, nor are they treated in a unified,
systematic way.  For example, it is common to apply circular rotations to chroma features 
in order to achieve key invariance when modeling chord quality~\cite{lee2008acoustic}.
Alternately, synthetic mixtures of monophonic instruments have been used 
to train polyphonic transcription engines~\cite{kirchhoff2012multi}.
At the other end of the spectrum, some authors leave the content unchanged and 
only modify labels during training, as exemplified by the \emph{target smearing} 
method of Ullrich~\etal\ for training structural boundary 
detectors~\cite{ullrich2014boundary}.

% finally, recent studies have investigated stability of models by evaluating on degraded signals:
%   but it's not clear that the degraded signals resemble the distribution of naturally occurring sounds
%   our goal is different: train on degraded signals, and evaluate on unmodified signals
%   
%   
Finally, recent studies have used degraded signals to evaluate the stability of
existing methods for MIR tasks.
The audio degradation toolbox was developed specifically for this purpose, and was used
to measure the impact of naturalistic deformations of audio on several tasks, including
beat tracking, score alignment, and chord recognition~\cite{mauch2013audio}.
Similarly, Sturm and Collins proposed the ``Kiki-Bouba Challenge'' as a way to determine
whether statistical models of musical concepts actually capture the defining
characteristics of a category (\eg, genre), or are over-fitting to spurious
correlations~\cite{sturmkiki}.

In both of the studies cited above, models are fit to unmodified data, and evaluated in
degraded conditions under the control of the experimenter.  
Data augmentation provides the converse of this setting: models are fit to degraded data, 
and evaluated on unmodified examples.  The distinction between the two approaches is
critical.  The former attempts to measure the robustness of a system under synthetic
conditions, while the latter attempts to improve robustness by \emph{training} under
synthetic conditions. Note that with data augmentation, the evaluation set is outside 
the control of the experimenter, so the resulting comparisons are unbiased with respect
to the underlying distribution from which the data are sampled.  While this does not
directly measure robustness of the resulting system, it has still been observed that data
augmentation can be used effectively to reduce generalization error in
practice~\cite{krizhevsky2012imagenet}.


\section{Data augmentation architecture}

% 1. because of the complex structure of annotations, we need to be careful
%   a. annotations aren't just track-level, but generally time-keyed
%   b. simple deformations can change annotations, such as chord labels or pitch
%   frequencies
%
% 2. why not use the audio degradation toolbox?
%   a. we'd like to be more extensible, support annotation-dependent deformations
%   b. support multiple annotations per-track
%   c. want the ability to embed history within the annotations for reproducibility
%   purposes
%   d. closer integration with python libraries for machine learning (eg theano)
%
% 3. we developed a generic, plugin-oriented architecture for doing musical data augmentation
%   a. hooks into JAMS, and inherits validation/schema.  
%   b. also makes it easy to modify all annotations for a given track in one shot.
%   c. allows the developer to register a deformation against different types of data
%       eg, a ``pitch-shift'' deformer implements an audio deformation, pitch annotation modification, and
%       chord/key manipulators
%   e. modules can be chained or skipped in a pipeline, similar to sklearn feature extractors
%       pipelines can be serialized, stored, shared, and reimplemented easily
%   f. full history of modification state is preserved within the output JAMS sandbox, so the results are
%   documented and reproducible
%       this includes all random state
%   g. deformations can be stochastic, and potentially generate infinite streams of randomized data
%       more generally, a deformer object can implement its own state transition logic
%       using python iterators makes this simple, self-contained, and memory-efficient

\cite{humphreyjams}

% Give a figure illustrating how deformers work
%   deformer 
%       generates states as a function of a jam
%       registers callbacks against annotation types
%   pipeline
%       chains deformers together
%   bypass
%       make a deformer optional
\subsection{Audio deformation}

\subsection{Annotation deformations}

\subsection{Pipelines}

\section{Example application: multi-instrument recognition}

\cite{bittner2014medleydb}

\subsection{Data augmentation}

The data augmentation pipeline consists of four stages:

\begin{description}
    \item[Pitch shift] by $n \in \{-1, 0, +1\}$ semitones.
    \item[Time stretch] by a factor of $f \in \left\{ 2^{-1/2}, 1.0, 2^{1/2}\right\}$.
    \item[Background noise] under four conditions: no noise,
        subway,\footnote{\url{https://www.freesound.org/people/jobro/sounds/112252/}}
        crowded concert hall,\footnote{\url{https://www.freesound.org/people/klankbeeld/sounds/171317/}}
        and night-time city noise.\footnote{\url{https://www.freesound.org/people/inkhorn/sounds/231870/}}
        The latter three were mixed with random weights drawn uniformly in $[0.1, 0.4]$.
    \item[Dynamic range compression] under three preset conditions drawn from the {Dolby E}
        standards~\cite{dolbyE}: none, \emph{speech},
        and \emph{music (standard)}.
\end{description}

Pitch-shift and time-stretch operations were performed by Rubberband~\cite{rubberband}, and dynamic range
compression was performed by sox~\cite{sox}.

Combining all stages of the pipeline produces {$3\times 3\times 4\times 3 = 108$} variants of each input track.  To
simplify the experiments, we only compare the cumulative effects of the above
augmentations.  This results in five training conditions of increasing complexity:
\begin{enumerate}
    \item No augmentation;
        \vspace{-.5\baselineskip}
    \item Pitch shift;
        \vspace{-.5\baselineskip}
    \item Pitch shift and time stretch;
        \vspace{-.5\baselineskip}
    \item Pitch shift, time stretch, and background noise;
        \vspace{-.5\baselineskip}
    \item All stages.
\end{enumerate}

\subsection{Acoustic model}

% Input features:
%   CQT at 36 bpo, ranging from C3 to C8 => 216 bins, 512-frame hop at 22KHz
%   sample patches of 44 frames ~= 1.02s
%   logamplitude clipped to -80dB
%
%   CQT features enable 2d-convolution for pitch-invariant feature extraction
%   log scaling gives us relative amplitude invariance
%
% convolutional model:
%   input: 216 x 44
%   layer 1: 24 filters of shape (13x9)
%       output dimension: 24 x (216 - 2*13+1) x (44 - 2 * 9 + 1)
%       relu
%   layer 2: 48 filters of shape (24x9x7)
%       relu
%   downsample 2x2 max-pooling
%   dense layer of size 96
%       relu
%       dropout = 0.5
%   dense layer for output
%       sigmoid nonlinearity
%       weight decay
%   output: n_classes

\cite{librosa}

\subsection{Results}

% Repeat x5:
%   randomly partition artists
%   train, validate, test for each model
%   compare per-track meanAP of tags

\cite{scikit-learn}

\section{Conclusion}

% For bibtex users:
\bibliography{refs}

\end{document}
